\chapter{Introduction}
Distributed denial of service (DDoS) attacks aim to prevent legitimate users from accessing a service by exhausting either the network's or the server's resources. Such attacks have been known since the early 1980s \cite{zargar}. In current times, a large number of devices that are connected to wide-area-networks exist due to dynamic and simple access to these networks. This poses a threat to the infrastructure of service providers on said networks, since DDoS attacks are often executed using a large number of "Zombies" \cite{zargar} \cite{kamboj}, which are computers that are infected and can be operated remotely without the owner's knowledge. DDoS attacks can be classified as a complex type of attack. This justifies the application of visualization techniques to make the interrelated events visible to a user such as a network operator \cite{yelizarov}.  This chapter motivates the need for a DDoS visualization tool that aids in understanding an attack's pattern and internal structures by visualizing interrelated events that would otherwise be hard to be identified by a human operator.

\section{Motivation}
Since the advent of DDoS attacks becoming a viable tool to sabotage targets, said attacks ave changed in strategy and scale. Nowadays, attackers usually use a large network of remotely controlled computing devices to attack the target. These devices are often recruited through other attacks such as the usage of worms and trojans and are called "Zombies" or "botnets".
Flooding attacks, where an attacker overwhelms its target with requests, have caused significant financial losses since the late 1990s. The targets of such attacks will often include companies with larger resources and reliance of networks such as Yahoo, PayPal, Visa, to only name a few. The motivation of an attacker to launch a DDoS attack on a company or piece of infrastructure can be manifold. Reasons include financial gain, revenge, intellectual challenges and cyberwarfare.

In the early days of DDoS attacks, even individuals with limited knowledge of networks were able to launch attacks. This is where recent advancements in the defense against DoS attacks has proven to be successful. Although such simple, small-scale attacks can be mitigated nowadays, one can observe that DDoS attacks have increased in their frequency and capability\cite{zargar}.
This allows for the classification of a DDoS attack as complex, since many entities execute it, causing a large number of events at the target site\cite{yelizarov}.
To defend against the previously described attacks, different mechanisms and different classifications of mechanisms exist. For example, a mechanism can be classified based on deployment location, type of attack, target of attack and point in time where the mechanism is applied \cite{zargar}.

An integral part in the defense of such attacks lies in its detection. Usually, an administrator would try to gather data about the network to understand its state. This is a difficult step since this data is often of large volume and contains many dimensions.
Since this data is still interpreted manually by a human operator, the application of data visualization techniques can ease the interpretation of such data, since humans find the interpretation of images easier\cite{wu}.

\section{Description of Work}
This work describes the conceptualization, architecture, design and implementation of a visualization application for DDoS attacks.
The DDoS attacks are visualized from network traffic logs that have been collected or generated by a network administrator. This system is designed to visualize attacks after they have been executed and possibly even after they have been detected and mitigated. The purpose of such a system does not lie in real-time detection and mitigation. The purpose of the system spans the understanding of recent attacks for educational purposes and for strategic decisions. For example, a researcher might want to understand a novel kind of attack, a student may want to learn about a specific type of attack and a network operator might want to improve his infrastructure against future attacks.

This allows for a classification of the system that is being built. Following the taxonomy outlined by Zargar et al, this visualization system can be applied to both network- and application-level attacks \cite{zargar}. This is possible since many network capturing applications allow precise filtering for different protocols in different layers during capturing \cite{tcpdump}. This information can then be used in the visualization of the captured information.
A network-level attack aims at the capacity of the overall system, for example by manually congestion caused by flooding datagrams. In an application-level attack, the specifics of the targeted application may be known and exploited. For example, an attacker might send a high-workload request, such as a HTTP request for a website displaying search results.
Defense mechanisms can be applied at various locations such as the source, destination, network or at hybrids of the former.
The visualization system that is being developed in this project is agnostic to the location where the packets are captured and thus supports all four classifications.

A second classification can be made with respect to the point in time where an attack can be defended against. These include before, during and after an attack.
Although our application can visualize network state before an attack, it would presumably prove most useful for traffic captured during or after an attack\cite{zargar}.

As has been stated above, the application visualizes captured traffic in formats such as "PCAP" files, which are used by many popular packet capturing applications such as tcpdump and Wireshark. As a prototype, this application showcases this feature by visualizing attacks contained in a database called ddosdb.org. This database has been set up for educational purposes to gather insights into attacks and targets academia and security professionals. Attack data contains full network traces and a unique attack fingerprint\cite{ddosdb}. Our application shows a prototypical implementation that visualizes both pieces of data. However, the architecture of the application should remain flexible in order to be extended with different data sets, models and visualization methods.

\section{Thesis Outline}

This chapter gave an introduction into why this application has been developed and what the scope of this work is. The remainder of this report is structured as follows: In chapter two, the current state in research is shown. This includes the current state of DDoS attacks and defense mechanisms, especially visualization systems that are part of a defense mechanism. Chapter three first outlines the efforts taken towards gaining an understanding of the problem domain and requirements to a DDoS visualization system. Finally, it introduces how we propose to solve the problem. In chapter four follows a section on the design and implementation of the application through a prototypical implementation of our approach. The main goal of this work is to prove the feasibility and added value of the applied approach. An evaluation in chapter five and a summary in chapter six conclude this report.