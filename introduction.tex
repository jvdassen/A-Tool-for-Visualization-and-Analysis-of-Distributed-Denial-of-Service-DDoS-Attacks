\chapter{Introduction}
DDoS attacks aim to prevent a legitimate user from accessing a service by exhausting either the networks or the servers resources. Such attacks have been known since the early 1980s \cite{zargar}. In current times a large number of devices that are connected to wide-area-networks exist due to dynamic and simple access to these networks. This poses a threat to the infrastructure of service providers on said networks since DDoS attacks are often executed using a large number of "Zombies" \cite{zargar} \cite{kamboj}. DDoS attacks can be classified as a complex type of attack. This justifies the application of visualization techniques to make the interrelated events visible to a user such as for example a network operator \cite{yelizarov}.  This chapter motivates the need for a DDoS visualization tool that aids in understanding an attacks pattern and internal structure by visualizing interrelated events that would otherwise be hard to be identified by a human operator.

\section{Motivation}
DDoS attacks, which have been known since the 1980s, have changed in strategy and scale. Nowadays, attackers usually use a large network of remotely controlled computing devices to attack the target. These devices are often recruited through other attacks such as the usage of worms and trojans and are called "Zombies" or "botnets".
Flooding attacks, where an attacker overwhelms its target with request, have caused significant financial losses since the late 1990s. The targets of such an attack are often even (??) companies with larger resources such as Yahoo, PayPal, Visa to only name a few. The motivation of an attacker to launch a DDoS attack on a company or piece of infrastructure can be manifold. Reasons include financial gain, revenge, intellectual challenges and cyperwarfe.

In the early days, even individuals with limited knowledge were able to launch attacks. This is where recent advancements in the defense against DoS attacks has proven to be successfull. Although such simple, small-scale attacks can be mitigated nowadays, one can observe that DDoS attacks have increased in their frequency and capability\cite{zargar}.
This allows for the classification of a DDoS attack as a complex attack, since many entities execute the attack, causing a large number of events at the target site\cite{yelizarov}.
To defend against the previously described attacks different mechanisms and different classifications of mechanisms exist. For example, a mechanism can be classified based on deployment location, type of attack, target of attack and point in time where the mechanism is applied \cite{zargar}.

An integral part in the defense of such attacks lies in its detection. Usually, an administrator would try to gather data about the network to understand its state. This is a difficult step since this data is often of large volume and contains many dimensions.
Since this data is still interpreted manually by a human operator, the application of data visualization techniques can ease the interpretation of such data, since humans find the interpretation of images easier\cite{wu}.


\section{Description of Work}
This work describes the conceptualization, architecture, design and implementation of a visualization application for DDoS attacks.
The DDoS attacks are visualized from network traffic logs that have been collected or generated by a network administrator. This system is designed to visualize attacks after they have been executed and possibly even after they have been detected and mitigated. The purpose of such a system does not lie in realtime detection and mitigation. 
after the attack

\section{Thesis Outline}

