\chapter{Summary, Conclusions, and Future Work} % MF: Summary, Conclusions, and Future Work 
\section{Summary and Conclusions}
The overarching goal of this work was the research on DDoS visualization systems and the design and implementation of such a system which allows attack visualization given a PCAP file as input. The first chapter motivated the need for such a system and described the scope of what was to be implemented. Next, related work was explored to gain an understanding of the relevant context. This included DDoS attacks, visualization systems for these attacks and research efforts aiming at providing capture files of such attacks for the research community.
Chapter \ref{sec:usecases} first outlined the exploration of available data and tools, followed by the requirements that were derived to the system to be built. With the problem domain defined, we proposed our solution by outlining an architecture.
The next chapter then described the efforts made towards implementing such a system and proving its feasibility and value. There, the design chosen for the system and the issues faced while implementing it were explained.
Finally, a chapter where the implemented solution is evaluated against three stakeholders concludes the report.

DDoSGrid, the solution developed as subject of this master project, is a fully functional platform for the mining of PCAP files so that DDoS attacks can be visualized. We see the value provided by this system in that it is open for extension and in that it does not focus on a specific DDoS attack type, which should allow for application in diverse contexts. The three stakeholders, researcher, network operator and teacher, that were used as a focal point of this thesis can all leverage this application for their day to day work.

\section{Future Work}
As future work of this solution we see the improvements and extensions outlined in section \ref{limitations}. The main areas of future work are thus the integration of new data sources and data processing. Next, the extension using new protocol parsers and feature extractors would conclude the future work of the backend modules. For the front end the future work consists of improvements such as user management, display and capture filtering as well as additional visualizations.

The prototypical implementation described in this paper resembles a general purpose analysis and visualization system. An interesting work would be to use this platform as a basis to investigate a specific attack or technology and determine its fit for the application in such a context.