\chapter{Related Work}

\section{DDoS attacks}
In a distributed Denial of Service attack, a malicious actor uses a network of controlled nodes (botnet) to target a specific host with repeated network traffic and requests. This network load can have many effect on the target's infrastructure, such as reduced capacity for new incoming honest nodes, slowed down response times or even total network failure. Challenges in combating these types of attacks is the identification and distinction from regular connections, as well as visualizing the threats to a network administrator.
\section{Visualisation Systems}
Yelizarov et al.\cite{yelizarov} presented a solution to visualize complex attacks on a network in 3D space, using visual patterns that should abstract long log files and journals of an attack's event. A complex attack differs from a simple attack, in that it consists of a large set of simple attacks that are interrelated, due to being planned and purposefully launched by the same entity that controls a large amounts of nodes. The data they used was already preprocessed and categorized by the intrusion detection system (IDS). In a simulated environment the authors targeted a network, in order to generate data that could be visualized. Different 3D environments were defined, such as cartesian planes where the x and y-axis represented time and internal host, respectively. The broken down network events are visualized by cylindrical pillars, that differ in height, color, thickness and position to each other, based on the event's metrics. On a second plane, the external hosts (attackers) were also mapped and connected to their corresponding network events, to visualize what parts of the system were targeted by whom and in what manner. Using their method, interrelations between simple attacks that were part of a complex attack are visualized and presented to a network admin for easier identification of a distributed attack.

In a literature survey conducted by Wu et al.\cite{wu}, the authors classified existing visualization methods in a fashion inspired by Shiravi et al.\cite{shiravi}, which is a use-case driven approach. The chosen use-cases were \textit{Network State Monitoring}, \textit{Traffic Flow Montoring} and \textit{Attack Patterns}. A frequent pattern in \textit{Network State Monitoring} visualizations was the use of a timeline to display the usage statistics of the host being attacked, such as network throughput and CPU usage\cite{chen, lee}. Another approach by Yang and Liu\cite{yang} distinguished honest from attacking nodes connected to the host. The distinction works by measuring resource consumption of the nodes over a one minute interval and then applying a K-Means algorithm to determine if a node is honest. 
In the \textit{Traffic Flow Monitoring} use case, a recurring theme was visualizing the nodes in a radial arrangement\cite{zhangRadial, pearlman, barbosa}, where the host was at the center and the connected nodes were spread around the host resulting in a ring-like formation, if enough nodes were connected. Zhang et al.\cite{zhangRadial} managed to make dishonest nodes stand out by setting the distance of a node to the host depending on the communication frequency between them. Additionally the color of a node becomes more visually striking, the more ports it calls on the host. Other solutions include tooltips on nodes that display more information about them, as well as filters to set a desired quantity of traffic and traffic load\cite{barbosa}. Another popular visualization is to cluster nodes together that have similar traits such as behavior and IP addresses\cite{zhangCluster}. This way identifying potentially malicious actors becomes possible by eye.
For the \textit{Attack Pattern} use-case, the high-dimensional attack-data is matched to pre-defined patterns to inform the observer about a potential attack and its characteristics\cite{wu}. Parallel coordinate systems were often used to categorize the attack data into distinct attributes such as source IP-address or targeted port\cite{choi, lee, okada}. With this method, recurring patterns can be recognized and identified as attacks.


\section{DDoSDB}\label{ddosdb}
DDoSDB is an open-source project that has the purpose of sharing DDoS attacks. The objective of sharing such attacks is that other parties such as security professionals and academia can study and compare these attacks. This should allow for the improvement of DDoS attack detection and mitigation\cite{ddosdb}.
The DDoSDB which has been developed at the University of Twente offers the following features:

\subsection{Database}
    
DDoSDB provides a number of indexed attacks which have been provided from victims of such attacks. The data is anonymized and contains a fingerprint of each attack as well as a full network capture in the form of a pcap file. \cite{ddos_dissector}
The database is stored in an ElasticSearch instance to support filtering, searching and sorting in the front-end application \cite{ddosdb-help}.

\subsection{ddos\_dissector}
    
The project provides a CLI tool that can generate an attack fingerprint from a network trace. The input file may be in different file formats, such as pcap or netflow files.
Output includes the aforementioned fingerprint as well as an anonymised network trace. Finally, one may upload processed data using this tool to the database if sufficient permissions are given(?) \cite{ddos_dissector}.

A fingerprint contains characterizing information about an attack such as protocol, source IP addresses with country and corresponding AS network, ports used and aggregated metrics such as attack durations, number of ip addresses\cite{ddosdb}.

\subsection{Web Frontend}
The project provides a web application written in the Django framework. This front-end lets users search for attacks, inspect fingerprints and compare IP addresses in a selection of attacks.
The search functionality directly supports the Lucene query language and therefore allows complex queries on the dataset in the ElasticSearch database.
The comparison feature can be considered a visualization since it allows the comparison of multiple attack vectors in a table. There, it will show for all combinations of selected attacks how similar the attack vectors are by source IP. It will render the percentage in the table cell and also colorcode the similarity by changing the transparency of the background color of a cell\cite{ddosdb}.

