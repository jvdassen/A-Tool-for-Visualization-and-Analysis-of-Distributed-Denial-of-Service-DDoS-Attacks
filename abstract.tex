\chapter*{Abstract}
\addcontentsline{toc}{chapter}{Abstract}

\selectlanguage{german}
Da DDoS Attacken in den letzten Jahren umfassender und heftiger wurden, wird der Bedarf nach Analysen immer notwendiger um einerseits Entwicklungen in Attackenstrategien aufzudecken und andererseits präventativ existierende Netzwerke davor zu schützen. Jedoch sind viele verfügbare Analysesysteme entweder geschlossen, kommerzieller natur oder Teil einer fixen Netzwerkinstallation welches nur jenes spezifische Netzwerk analysiert.

Diese Arbeit dokumentiert die Konzeptualisierung und Entwicklung einer allgemeinen, Forschungs-orientierten Applikation, welche Benutzer Netzwerkaufnahmen analysieren und visualisieren lässt. Diese Aufnahmen können der Applikation bereitgestellt werden, welche Metriken berechnet und Merkmale aus dem Datenset extrahiert. Diese analysierten Daten können dann visualisiert werden sodass Erkenntnisse daraus gezogen werden können. Visualisierungen und Datensets werden auf einem konfigurierbaren Raster dargestellt, sodass Benutzer die vorhandenen Informationen optimal für ihren Anwendungsfall arrangieren können. Es wurde Fokus gelegt auf die Möglichkeit die Applikation zu erweitern, indem interessierte ihre eigenen Merkmalextrahierer und Visualisierungen programmieren können. Die Applikation wurde evaluiert indem drei Anwendungsfälle, basierend auf Interessenvertreter (\emph{Forscher}, \emph{Netzwerkbetreiber} und \emph{Professor}) dokumentiert wurden, welche alle spezifische Interessen und Anforderungen and die konzipierte Anwendung haben.




\selectlanguage{english}

With DDoS attacks becoming more widespread and severe in recent years, the analysis of such attacks has become necessary from both a research and network operational standpoint in order to understand current trends in such attacks and preemptively protect networks from them. However, many solutions that allow such analyses are either closed source, sold commercially or part of a larger network infrastructure whose scope only includes that specific network. 

This paper documents the conceptualization and development of a general-purpose, research oriented application that lets end-users analyze and visualize network capture data. This capture data can be supplied to the application, which in turn calculates metrics and extracts features of that data set. This analyzed data can then be visualized and interacted with. Visualizations and data sets can be arranged on a customizable grid that lets users configure dashboards to serve their needs of gathering information about a DDoS attack. Focus was set on the capability of extending the application by letting interested parties write their own feature extractors and visualizations. The application was evaluated by documenting three case studies based off stakeholders (\emph{Researcher}, \emph{Network Operator} and \emph{Professor}) that all have distinctive requirements and interests regarding such an application.